\documentclass{SHUarticle}
\title{\heiti{决策树算法总结}}
\date{\today}
\keyword{ID3\quad  C4.5\quad CART\quad  决策树减枝\quad 递归算法\quad }
\begin{document}
	\maketitle
	\begin{cnabstract}
		决策树是一种十分常见的分类和回归算法,决策树学习是从训练集中归纳出一种分类规则,并以树状图形的方式
		表现出来,可以看成为一个if-then规则的集合。决策树分类通常包含ID3、C4.5、CART三种算法,三者选择最优特征的方法有所不同:
		其中ID3算法使用的是信息增益;C4.5算法使用的是信息增益比;CART算法使用的是基尼指数。
		三种算法适用的场景也各不相同:
		\begin{itemize}
			\item ID3算法:适用于离散型数据。ID3算法使用信息增益来选择最佳的分裂变量,
			因此需要将数据集离散化为有限的分类。因此,
			ID3算法通常用于文本分类、垃圾邮件分类等离散型数据的分类问题。
			\item C4.5算法:适用于离散型和连续型数据。与ID3算法不同,
			C4.5算法使用信息增益比来选择最佳的分裂变量,能够处理连续型和离散型的数据,因此适用范围更广。C4.5算法常用于数据挖掘、信用评分等领域。
			\item  CART算法:适用于离散型和连续型数据。与ID3算法不同,
			C4.5算法使用信息增益比来选择最佳的分裂变量,
			能够处理连续型和离散型的数据,因此适用范围更广。
			C4.5算法常用于数据挖掘、信用评分等领域。
		\end{itemize}
	\end{cnabstract}
\section{三种算法的优缺点}
\subsection{三种算法的优点}
\begin{itemize}
	\item ID3算法:算法简单易懂,计算效率高。
	对于数据缺失的情况有很好的处理能力。
	\item C4.5算法:
	\begin{itemize}
		\item 支持离散型和连续型数据的处理。
		\item  采用信息增益比来选择最佳分裂变量,能够避免ID3算法中选择取值较多的属性作为分裂变量的问题。
		\item 采用剪枝操作,能够有效地防止过拟合。
	\end{itemize}
	\item CART算法:
	\begin{itemize}
		\item 能够处理离散型和连续型数据的分类和回归问题。
		\item 采用基尼指数来选择最佳分裂变量,能够更好地处理连续型数据
		\item 采用剪枝操作,能够有效地防止过拟合.
	\end{itemize}
\end{itemize}
\subsubsection{三种算法的缺点}
\begin{itemize}
	\item ID3算法:
	\begin{itemize}
		\item 对于连续型数据和缺失数据的处理能力不足。
		\item 容易产生过拟合,不能很好地应对噪声数据。
	\end{itemize}
	\item C4.5算法:
	\begin{itemize}
		\item 对于噪声数据的处理能力不足。
		\item 计算效率较低,需要对数据进行多次扫描。
	\end{itemize}
	\item CART算法:
	\begin{itemize}
		\item CART算法生成的是二叉树,对于多分类问题需要进行二次划分,增加了计算复杂度。
		\item 对于缺失数据的处理能力有限。
	\end{itemize}
\end{itemize}
\section{算法框架的使用}
 \begin{algorithm}
	\caption{ID3算法}
	\label{algo:ref}
	\begin{algorithmic}[1]
		\REQUIRE 训练数据集D,特征集A.  % this command shows "Input"
		\ENSURE ~\\           % this command shows "Initialized"
		some text goes here ... \\
		\WHILE {\emph{not converged}}
		\STATE ... \\  % line number at left side
		\ENDWHILE
		\RETURN 决策树T.  % this command shows "Output"
	\end{algorithmic}
\end{algorithm}
\section{数学定理定义的使用}
\begin{theorem}
	 设$a,b$是两个实数,则$2ab\leq a^2+b^2$
\end{theorem}
\begin{proof}
	因为$(a-b)^{2}\geq 0$\\
	所以可得到$a^{2}+b^{2}-2ab\geq 0$,从而得到$2ab\leq a^2+b^2$。
\end{proof}
\begin{lemma}
	引理1
\end{lemma}
\begin{proposition}
	命题一
\end{proposition}
\end{document}